% geni_4.tex - GENI Lab 4 for Cloud Computing class (Spring 2015)
% Chanmann Lim - April 2015

\documentclass[a4paper]{article}

\usepackage[margin=1 in]{geometry}
\usepackage{listings}
\usepackage{graphicx}
\usepackage{float}

\begin{document}
\title{CS 7001-03: Report for Lab 4: InterCloud Web Services for OpenStack\-based Cloud Orchestration}
\author{Chanmann Lim\\ 
	\texttt{cl9p8@mail.mail.missouri.edu}}
\date{April 21, 2015}
\maketitle

% ---------------------------------------- 1 ----------------------------------------
\paragraph{1. } Provide screenshot of the "Network Topology" after a new instance is created. Explain the graph. 

% ---------------------------------------- 2 ----------------------------------------
\paragraph{2. } Provide screenshot of the 'controller' node with the MAC Address clearly displayed.


% ---------------------------------------- 3 ----------------------------------------
\paragraph{3. } List in detail the resources available for the deployed cloud infrastructure (vCPUs, RAM, Floating IPs, Security Groups, and Volumes)

% ---------------------------------------- 4 ----------------------------------------
\paragraph{4. } List the necessary changes in the profile file to add an extra compute node, and submit a revised RSpec.

% ---------------------------------------- 5 ----------------------------------------
\paragraph{5. } Extend the Intercloud API to display user list (KEYSTONE) as:
curl -u clouduser:EasyPassword15 \-i http://[IP]:8090/list\_user Provide screenshot of the output.

% ---------------------------------------- 5 ----------------------------------------
\paragraph{6. } By using your AWS instance setup in AWS Lab\-2, you should write a web service client (use any language of your preference) to request and display the cloud information available in the JSON file in a simple web site. Include the Amazon DNS link and the code in your submission report.

\end{document}