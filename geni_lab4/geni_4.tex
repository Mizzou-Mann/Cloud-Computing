% geni_4.tex - GENI Lab 4 for Cloud Computing class (Spring 2015)
% Chanmann Lim - April 2015

\documentclass[a4paper]{article}

\usepackage[margin=1 in]{geometry}
\usepackage{listings}
\usepackage{graphicx}
\usepackage{float}

\begin{document}
\title{CS 7001-03: Report for Lab 4: InterCloud Web Services for OpenStack\-based Cloud Orchestration}
\author{Chanmann Lim\\ 
	\texttt{cl9p8@mail.mail.missouri.edu}}
\date{April 21, 2015}
\maketitle

% ---------------------------------------- 1 ----------------------------------------
\paragraph{1. } Screenshot of the "Network Topology" in CloudLab: \\
\begin{figure}[H]
  \centering
    \includegraphics[scale=.37]{net_topo.png}
  \caption{Network Topology in CloudLab}
\end{figure}

The "Network Topology" tab under the Network section in the above Figure shows the three networks represented by three columns in different colors connected with each others through two routers and a newly launched instance connected to two networks namely "flat-data-net" and "tun-data-net". 

% ---------------------------------------- 2 ----------------------------------------
\paragraph{2. } Screenshot of the "controller" node's MAC address: \\
\begin{figure}[H]
  \centering
    \includegraphics[scale=.54]{mac_address.png}
  \caption{Controller node's MAC address}
\end{figure}

% ---------------------------------------- 3 ----------------------------------------
\paragraph{3. } The available resources of the deployed cloud infrastructure are summarized in the "Overview" tab under "Project \textgreater{} Compute" section: \\
\begin{description}
\leftskip 0.4in
\parindent -0.4in
	\item[vCPUs: ] 20.
	\item[RAM: ] 50GB.
	\item[Floating IPs: ] 50.
	\item[Security Groups: ] 10.
	\item[Volumes: ] 10.
	\item[Volume Storage: ] 1000GB.
	\item[Instances: ] 10.
\end{description}

% ---------------------------------------- 4 ----------------------------------------
\paragraph{4. } To add an extra compute node to the "Tutorial-OpenStack" profile, we need to copy it to create a new profile then change the "rspec" as follow: \\

\noindent Add compute node 2:
\small{
\begin{verbatim}
<node client_id="compute2">
    <sliver_type name="raw">
      <disk_image name="urn:publicid:IDN+utah.cloudlab.us+image+emulab-ops//UBUNTU14-10-64-OSCPF"/>
    </sliver_type>
    <services>
      <execute shell="sh" command="sudo /tmp/setup/setup-driver.sh"/>
      <install url="http://www.emulab.net/downloads/openstack-setup-v2.tar.gz" 
               install_path="/tmp"/>
    </services>
    <interface client_id="compute2:if0"/>
    <site xmlns="http://www.protogeni.net/resources/rspec/ext/jacks/1" id="28"/>
</node>
\end{verbatim}} ~\\


\noindent Change link tag to add 'compute2:if0' to the LAN:
\small{
\begin{verbatim}
<link client_id="lan-1">
    <link_type name="lan"/>
    <interface_ref client_id="controller:if0"/>
    <interface_ref client_id="networkmanager:if0"/>
    <interface_ref client_id="compute1:if0"/>
    <interface_ref client_id="compute2:if0"/>
</link>
\end{verbatim}

% ---------------------------------------- 5 ----------------------------------------
\paragraph{5. } The screenshot of list\_user RESTful API output:
\begin{figure}[H]
  \centering
    \includegraphics[scale=.44]{list_user.png}
  \caption{list\_user RESTful API}
\end{figure}

% ---------------------------------------- 5 ----------------------------------------
\paragraph{6. } By using your AWS instance setup in AWS Lab\-2, you should write a web service client (use any language of your preference) to request and display the cloud information available in the JSON file in a simple web site. Include the Amazon DNS link and the code in your submission report.

\end{document}